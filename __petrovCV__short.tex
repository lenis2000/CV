\documentclass[letterpaper,11pt]{article}
\usepackage{hyperref}
\usepackage{geometry,amsmath,amsfonts}
\usepackage[T1]{fontenc}
\usepackage[sc,osf]{mathpazo}
\usepackage{etaremune,enumerate}

\def\name{Leonid Petrov}
\def\footerlink{https://lpetrov.cc}
\allowdisplaybreaks
\usepackage{longtable}

\hypersetup{
  colorlinks = true,
  urlcolor = blue,
  pdfauthor = {\name},
  pdftitle = {\name: Leonid Petrov},
  pdfsubject = {Leonid Petrov},
  pdfpagemode = UseNone
}

\geometry{
  body={6.2in, 8.25in},
  left=1.2in,
  top=1.45in
}

% Customize page headers
\pagestyle{myheadings}
\markright{\name}
\thispagestyle{empty}

% Custom section fonts
\usepackage{sectsty}
\sectionfont{\rmfamily\mdseries\Large}
\subsectionfont{\rmfamily\mdseries\itshape\large}

% Other possible font commands include:
% \ttfamily for teletype,
% \sffamily for sans serif,
% \bfseries for bold,
% \scshape for small caps,
% \normalsize, \large, \Large, \LARGE sizes.

% Don't indent paragraphs.
\setlength\parindent{0em}

% Make lists without bullets
% \renewenvironment{itemize}{
%   \begin{list}{}{
%     \setlength{\leftmargin}{1.5em}
%   }
% }{
%   \end{list}
% }

\begin{document}


\centerline{\huge \bf \name . Brief CV}
\bigskip

\textbf{Full version is at \url{https://lpetrov.cc/research/petrovCV.pdf}}

\noindent\hrulefill

\vspace{0.25in}

Department of Mathematics, University of Virginia

\href{mailto:petrov@virginia.edu}{\tt petrov@virginia.edu}

\url{https://lpetrov.cc}

\url{https://orcid.org/0000-0003-4607-7399}

\section*{Research areas}

Probability, Mathematical Physics, Algebraic Combinatorics, Representation
Theory.

\section*{Education}

\begin{itemize}
	\item [2010:]
		Ph.D.,
		Institute for Information Transmission Problems.

	\item [2007:]
		Diploma,
		Lomonosov Moscow State University,
		Department of Mathematics and Mechanics (specialization in Probability).
\end{itemize}

\section*{Appointments}

\begin{itemize}
	\item
	[Since 2024:]
	Professor\\ at Department of Mathematics, University
	of Virginia, Charlottesville, VA, USA.
\item
	[2019--2024:]
	Associate Professor\\ at Department of Mathematics, University
	of Virginia, Charlottesville, VA, USA.
	\item
	      [2014--2019:]
	      Assistant Professor\\ at Department of Mathematics, University
	      of Virginia, Charlottesville, VA, USA.
	\item
	      [2011--2014:]
	      Research Instructor\\ at Department of Mathematics, Northeastern
	      University, Boston, MA, USA.
	\item
	      [2009--2011:]
	      Research associate\\ at Dobrushin Mathematics Laboratory,
	      Institute for Information Transmission Problems, Moscow, Russia.
\end{itemize}

\section*{Visiting Appointments}

\begin{itemize}
	\item [Spring 2024:]
		  Senior Fellow
		  \\
		  IPAM Program ``Geometry, Statistical Mechanics, and Integrability''
	\item [Fall 2021:]
		  Research Professor
		  \\
		  MSRI Program ``Universality and Integrability in Random Matrix Theory and Interacting Particle Systems''
	\item
	      [2017--2018:]
	      Visiting Assistant Professor\\ at Department of Mathematics, MIT,
	      Cambridge, MA, USA.
\end{itemize}

\section*{Recent prizes/funding}
\begin{itemize}
	\item [2022--2026:]
	NSF DMS grant 2153869
	``Random Systems from Symmetric Functions and Vertex Models'',
	\$320,654.

	\item [2022--2024:]
	4-VA at UVA Collaborative Research Grant program
	``Randomness by algebraic structures'',
	\$30,000.


	\item [2020--2025:]
		Simons Collaboration Grant for Mathematicians 709055
		``Distributional symmetries in stochastic systems'',
		\$42,000.
	\item [2019:]
		The 2020 Bernoulli prize for an outstanding survey article in probability
		(jointly with Alexei Borodin for the paper \emph{Integrable probability: From representation theory to
		Macdonald processes})
	\item
		[2018-2019:] PI,
				NSF DMS conference grant
				1839534
				``Workshop on Representation Theory, Combinatorics, and Geometry'',
				amount \$15,000.


	\item
	      [2017:] Simons Foundation Collaboration Grant for
	      Mathematicians. Recommended for funding but not awarded due to
	      the receipt of the NSF DMS grant 1664617 (as per the rules of Collaboration
	      Grants).
	\item
			[2017--2022:] PI, NSF DMS grant 1664617
	      ``FRG: Collaborative Research: Integrable Probability''.
	      Joint with PIs Jinho Baik (University of Michigan), Alexei
	      Borodin, Vadim Gorin (MIT), and Ivan Corwin (Columbia University). Amount:
	      \$193,453 (UVA part).
	\item
		[2016--2017:] Co-PI,
	      NSF DMS conference grant 1663552 ``2017 Seminar on Stochastic
				Processes''.
	% \item
	%       [2015:] Prize of the Moscow Mathematical Society.
	% \item
	%       [2014--2015:] EDF Fellowship of the University of Virginia.
	% \item
	%       [2014:] AMS/NSF Travel Grant Award for ICM 2014.
%	\item
%	      [2011--2013:] RFBR--CNRS grant 11-01-93105 ``Representation
%	      theory and noncommutative geometry''.
%	\item
%	      [2010--2012:] RFBR--CNRS grant 10-01-93114 ``New models of
%	      Markov processes on point configurations. Applications to
%	      stochastic queueing networks''.
%	\item
%	      [2010:] Dynasty foundation fellowship for young scientists.
%	\item
%	      [2010:] Silver prize of The Fourteenth M\"obius Contest.
%	\item
%	      [2009:] Alexander Kuznetsov/Independent University of Moscow
%	      graduate student scholarship.
%	\item
%	      [2005, 2006:] V. Potanin federal scholarship for academic
%	      excellence, leadership and creativity.
\end{itemize}

\section*{Publications ($*$ --- preprints)}

\begin{etaremune}
\renewcommand{\labelenumi}{[\theenumi]}




\item[{[48]}] ($*$)
Alexey Bufetov, Leonid Petrov, Panagiotis Zografos.
\emph{Domino Tilings of the Aztec Diamond in Random Environment and Schur Generating Functions}, 
\href{https://arxiv.org/abs/2507.08560}{\texttt{arXiv:2507.08560 [math.PR]}}.















\item[{[47]}] ($*$)
Leonid Petrov, Jeanne Scott.
\emph{Random Fibonacci Words via Clone Schur Functions}, 
\href{https://arxiv.org/abs/2412.21126}{\texttt{arXiv:2412.21126 [math.PR]}}.









\item[{[46]}] ($*$)
Greta Panova, Leonid Petrov.
\emph{Hook-length Formulas for Skew Shapes via Contour Integrals and Vertex Models}, 
\href{https://arxiv.org/abs/2409.17842}{\texttt{arXiv:2409.17842 [math.CO]}}.







\item[{[45]}] ($*$)
Alejandro H. Morales, Greta Panova, Leonid Petrov, Damir Yeliussizov.
\emph{Grothendieck Shenanigans: Permutons from Pipe Dreams via Integrable Probability}, Accepted at FPSAC 2025 as a poster. Full version submitted to a journal. 
\href{https://arxiv.org/abs/2407.21653}{\texttt{arXiv:2407.21653 [math.PR]}}.











\item[{[44]}] 
Amol Aggarwal, Matthew Nicoletti, Leonid Petrov.
\emph{Colored Interacting Particle Systems on the Ring: Stationary Measures from Yang--Baxter Equation}, Compositio Math., to appear. 
\href{https://arxiv.org/abs/2309.11865}{\texttt{arXiv:2309.11865 [math.PR]}}.









\item[{[43]}] 
Svetlana Gavrilova, Leonid Petrov.
\emph{Tilted biorthogonal ensembles, Grothendieck random partitions, and determinantal tests}, Selecta Math. (2024), Volume 30, article 56. 
\href{https://arxiv.org/abs/2305.17747}{\texttt{arXiv:2305.17747 [math.PR]}}.







\item[{[42]}] 
Leonid Petrov, Mikhail Tikhonov.
\emph{Asymptotics of noncolliding q-exchangeable random walks}, J. Phys. A: Math. Theor. 56 365203. 
\href{https://arxiv.org/abs/2303.02380}{\texttt{arXiv:2303.02380 [math.PR]}}.





\item[{[41]}] 
Leonid Petrov, Axel Saenz.
\emph{Rewriting History in Integrable Stochastic Particle Systems}, Commun. Math. Phys., 405 (300), 2024. 
\href{https://arxiv.org/abs/2212.01643}{\texttt{arXiv:2212.01643 [math.PR]}}.









\item[{[40]}] 
Leonid Petrov.
\emph{Noncolliding Macdonald walks with an absorbing wall}, SIGMA 18 (2022), 079, 21 pages. 
\href{https://arxiv.org/abs/2204.09206}{\texttt{arXiv:2204.09206 [math.PR]}}.



\item[{[39]}] 
Matthew Nicoletti, Leonid Petrov.
\emph{Irreversible Markov Dynamics and Hydrodynamics for KPZ States in the Stochastic Six Vertex Model}, Electronic Journal of Probability 2023, Vol. 28, paper no. 138, 1-55.. 
\href{https://arxiv.org/abs/2201.12497}{\texttt{arXiv:2201.12497 [math.PR]}}.











\item[{[38]}] 
Amol Aggarwal, Alexei Borodin, Leonid Petrov, Michael Wheeler.
\emph{Free Fermion Six Vertex Model: Symmetric Functions and Random Domino Tilings}, Selecta Math., 29, article 36 (2023). 
\href{https://arxiv.org/abs/2109.06718}{\texttt{arXiv:2109.06718 [math.PR]}}.

















\item[{[37]}] 
Leonid Petrov.
\emph{Refined Cauchy identity for spin Hall-Littlewood symmetric rational functions}, Journal of Combinatorial Theory Ser. A, vol. 184 (2021), 105519. 
\href{https://arxiv.org/abs/2007.10886}{\texttt{arXiv:2007.10886 [math.CO]}}.







\item[{[36]}] 
Matteo Mucciconi, Leonid Petrov.
\emph{Spin q-Whittaker polynomials and deformed quantum Toda}, Communications in Mathematical Physics, 389, pages 1331-1416 (2022). 
\href{https://arxiv.org/abs/2003.14260}{\texttt{arXiv:2003.14260 [math.PR]}}.









\item[{[35]}] 
Leonid Petrov, Mikhail Tikhonov.
\emph{Parameter symmetry in perturbed GUE corners process and reflected drifted Brownian motions}, Journal of Statistical Physics, 181 (2020), 1996-2010. 
\href{https://arxiv.org/abs/1912.08671}{\texttt{arXiv:1912.08671 [math.PR]}}.



\item[{[34]}] 
Leonid Petrov.
\emph{Parameter permutation symmetry in particle systems and random polymers}, SIGMA 17 (2021), 021, 34 pages. 
\href{https://arxiv.org/abs/1912.06067}{\texttt{arXiv:1912.06067 [math.PR]}}.



\item[{[33]}] 
Leonid Petrov.
\emph{PushTASEP in inhomogeneous space}, Electronic Journal of Probability, vol. 25 (2020), paper no. 114. 
\href{https://arxiv.org/abs/1910.08994}{\texttt{arXiv:1910.08994 [math.PR]}}.









\item[{[32]}] 
Leonid Petrov, Axel Saenz.
\emph{Mapping TASEP back in time}, Probability Theory and Related Fields, 182, pages 481-530 (2022). 
\href{https://arxiv.org/abs/1907.09155}{\texttt{arXiv:1907.09155 [math.PR]}}.









\item[{[31]}] 
Alexey Bufetov, Matteo Mucciconi, Leonid Petrov.
\emph{Yang-Baxter random fields and stochastic vertex models}, Advances in Mathematics 388 (2021), 107865. 
\href{https://arxiv.org/abs/1905.06815}{\texttt{arXiv:1905.06815 [math.PR]}}.













\item[{[30]}] 
Ivan Corwin, Konstantin Matveev, Leonid Petrov.
\emph{The q-Hahn PushTASEP}, International Mathematics Research Notices (2019),  rnz106. 
\href{https://arxiv.org/abs/1811.06475}{\texttt{arXiv:1811.06475 [math.PR]}}.





\item[{[29]}] 
Alisa Knizel, Leonid Petrov, Axel Saenz.
\emph{Generalizations of TASEP in discrete and continuous inhomogeneous space}, Communications in Mathematical Physics 372 (2019), no. 3, pp 797-864. 
\href{https://arxiv.org/abs/1808.09855}{\texttt{arXiv:1808.09855 [math.PR]}}.











\item[{[28]}] 
Christian Gromoll, Mark Meckes, Leonid Petrov.
\emph{Quenched Central Limit Theorem in a Corner Growth Setting}, Electronic Communications in Probability (2018), Vol. 23, paper no. 101, 1-12. 
\href{https://arxiv.org/abs/1804.04222}{\texttt{arXiv:1804.04222 [math.PR]}}.













\item[{[27]}] 
Alexey Bufetov, Leonid Petrov.
\emph{Yang-Baxter field for spin Hall-Littlewood symmetric functions}, Forum of Mathematics Sigma 7 (2019), e39. 
\href{https://arxiv.org/abs/1712.04584}{\texttt{arXiv:1712.04584 [math.PR]}}.













\item[{[26]}] 
Michael Damron, Leonid Petrov, David Sivakoff.
\emph{Coarsening model on Zd with biased zero-energy flips and an exponential large deviation bound for ASEP}, Communications in Mathematical Physics 362 (2018), no. 1, 185-217. 
\href{https://arxiv.org/abs/1708.05806}{\texttt{arXiv:1708.05806 [math.PR]}}.

















\item[{[25]}] 
Sevak Mkrtchyan, Leonid Petrov.
\emph{GUE corners limit of q-distributed lozenge tilings}, Electronic Journal of Probability, Volume 22 (2017), paper no. 101, 24 pp. 
\href{https://arxiv.org/abs/1703.07503}{\texttt{arXiv:1703.07503 [math.PR]}}.



\item[{[24]}] 
Alexei Borodin, Leonid Petrov.
\emph{Inhomogeneous exponential jump model}, Probability Theory and Related Fields 172 (2018), 323-385. 
\href{https://arxiv.org/abs/1703.03857}{\texttt{arXiv:1703.03857 [math.PR]}}.

















\item[{[23]}] 
Daniel Orr, Leonid Petrov.
\emph{Stochastic higher spin six vertex model and q-TASEPs}, Advances in Mathematics 317 (2017), 473-525. 
\href{https://arxiv.org/abs/1610.10080}{\texttt{arXiv:1610.10080 [math.PR]}}.









\item[{[22]}] 
Vadim Gorin, Leonid Petrov.
\emph{Universality of local statistics for noncolliding random walks}, Annals of Probability (2019), Vol. 47, No. 5, 2686-2753. 
\href{https://arxiv.org/abs/1608.03243}{\texttt{arXiv:1608.03243 [math.PR]}}.



\item[{[21]}] 
Alexei Borodin, Leonid Petrov.
\emph{Lectures on Integrable probability: Stochastic vertex models and symmetric functions}, Lecture Notes of the Les Houches Summer School, Volume 104, July 2015. 
\href{https://arxiv.org/abs/1605.01349}{\texttt{arXiv:1605.01349 [math.PR]}}.



\item[{[20]}] 
Alexei Borodin, Leonid Petrov.
\emph{Higher spin six vertex model and symmetric rational functions}, Selecta Mathematica 24 (2018), no. 2, 751-874. 
\href{https://arxiv.org/abs/1601.05770}{\texttt{arXiv:1601.05770 [math.PR]}}.









\item[{[19]}] 
Konstantin Matveev, Leonid Petrov.
\emph{q-randomized Robinson-Schensted-Knuth correspondences and random polymers}, Annales de l'Institut Henri Poincare D: Combinatorics, Physics and their Interactions 4 (2017), no. 1, 1-123. 
\href{https://arxiv.org/abs/1504.00666}{\texttt{arXiv:1504.00666 [math.PR]}}.





\item[{[18]}] 
Ivan Corwin, Leonid Petrov.
\emph{Stochastic higher spin vertex models on the line}, Communications in Mathematical Physics 343 (2016), no. 2, 651-700. 
\href{https://arxiv.org/abs/1502.07374}{\texttt{arXiv:1502.07374 [math.PR]}}.













\item[{[17]}] 
Alexei Borodin, Ivan Corwin, Leonid Petrov, Tomohiro Sasamoto.
\emph{Spectral theory for interacting particle systems solvable by coordinate Bethe ansatz}, Communications in Mathematical Physics 339 (2015), no. 3, 1167-1245. 
\href{https://arxiv.org/abs/1407.8534}{\texttt{arXiv:1407.8534 [math-ph]}}.





\item[{[16]}] 
Alexey Bufetov, Leonid Petrov.
\emph{Law of Large Numbers for Infinite Random Matrices over a Finite Field}, Selecta Mathematica 21 (2015), no. 4, 1271-1338. 
\href{https://arxiv.org/abs/1402.1772}{\texttt{arXiv:1402.1772 [math.PR]}}.













\item[{[15]}] 
Alexei Borodin, Leonid Petrov.
\emph{Integrable probability: From representation theory to Macdonald processes}, Probability Surveys, 11 (2014), 1-58. 
\href{https://arxiv.org/abs/1310.8007}{\texttt{arXiv:1310.8007 [math-ph]}}.





\item[{[14]}] 
Alexei Borodin, Ivan Corwin, Leonid Petrov, Tomohiro Sasamoto.
\emph{Spectral theory for the q-Boson particle system}, Compositio Mathematica, 151 (2015), no. 1, 1-67. 
\href{https://arxiv.org/abs/1308.3475}{\texttt{arXiv:1308.3475 [math-ph]}}.



\item[{[13]}] 
Ivan Corwin, Leonid Petrov.
\emph{The q-PushASEP: A New Integrable Model for Traffic in 1+1 Dimension}, Journal of Statistical Physics, 160 (2015), no. 4, 1005-1026. 
\href{https://arxiv.org/abs/1308.3124}{\texttt{arXiv:1308.3124 [math.PR]}}.





\item[{[12]}] 
Alexei Borodin, Leonid Petrov.
\emph{Nearest neighbor Markov dynamics on Macdonald processes}, Advances in Mathematics, 300 (2016), 71-155. 
\href{https://arxiv.org/abs/1305.5501}{\texttt{arXiv:1305.5501 [math.PR]}}.









\item[{[11]}] 
Leonid Petrov.
\emph{The Boundary of the Gelfand-Tsetlin Graph: New Proof of Borodin-Olshanski's Formula, and its q-analogue}, Moscow Mathematical Journal, 14 (2014) no. 1, 121-160. 
\href{https://arxiv.org/abs/1208.3443}{\texttt{arXiv:1208.3443 [math.CO]}}.



\item[{[10]}] 
Leonid Petrov.
\emph{Asymptotics of uniformly random lozenge tilings of polygons. Gaussian free field}, Annals of Probability, 43 (2014), no. 1, 1-43. 
\href{https://arxiv.org/abs/1206.5123}{\texttt{arXiv:1206.5123 [math.PR]}}.





\item[{[9]}] 
Leonid Petrov.
\emph{Asymptotics of Random Lozenge Tilings via Gelfand-Tsetlin Schemes}, Probability Theory and Related Fields, 160 (2014), no. 3, 429-487. 
\href{https://arxiv.org/abs/1202.3901}{\texttt{arXiv:1202.3901 [math.PR]}}.







\item[{[8]}] 
Leonid Petrov.
\emph{sl(2) Operators and Markov Processes on Branching Graphs}, Journal of Algebraic Combinatorics 38 (2013), no. 3, 663-720. 
\href{https://arxiv.org/abs/1111.3399}{\texttt{arXiv:1111.3399 [math.CO]}}.







\item[{[7]}] 
Leonid Petrov.
\emph{On Measures on Partitions Arising in Harmonic Analysis for Linear and Projective Characters of the Infinite Symmetric Group}, Proceedings of the international conference "50 years of IITP". 
\href{https://arxiv.org/abs/1107.0676}{\texttt{arXiv:1107.0676 [math.CO]}}.





\item[{[6]}] 
Leonid Petrov.
\emph{Pfaffian Stochastic Dynamics of Strict Partitions}, Electronic Journal of Probability 16 (2011), 2246-2295. 
\href{https://arxiv.org/abs/1011.3329}{\texttt{arXiv:1011.3329 [math.PR]}}.



\item[{[5]}] 
Leonid Petrov.
\emph{Random Strict Partitions and Determinantal Point Processes}, Electronic Communications in Probability 15 (2010), 162-175. 
\href{https://arxiv.org/abs/1002.2714}{\texttt{arXiv:1002.2714 [math.PR]}}.







\item[{[4]}] 
Leonid Petrov.
\emph{Random Walks on Strict Partitions}, Journal of Mathematical Sciences 168 (2010), no. 3, 437-463. 
\href{https://arxiv.org/abs/0904.1823}{\texttt{arXiv:0904.1823 [math.PR]}}.



\item[{[3]}] 
Leonid Petrov.
\emph{Limit Behavior of Certain Random Walks on Strict Partitions}, Russian Mathematical Surveys 64 (2009), no. 6, 1139-1141. 




\item[{[2]}] 
Leonid Petrov.
\emph{A Two-parameter Family of Infinite-dimensional Diffusions in the Kingman Simplex}, Functional Analysis and Its Applications 43 (2009), no. 4, 279-296. 
\href{https://arxiv.org/abs/0708.1930}{\texttt{arXiv:0708.1930 [math.PR]}}.



\item[{[1]}] 
Leonid Petrov.
\emph{Asymptotic Behavior of a Certain Collection of Particles on a Line Under Synchronization}, Proceedings of the XXVIII Conference of Young Scientists of Department of Mechanics and Mathematics of the Lomonosov Moscow State University (2006), 152-156, in Russian. 



\end{etaremune}



\subsection*{Other works}

\begin{etaremune}
	\renewcommand{\labelenumi}{[\theenumi]}
	\item
	Sihan Li, Andrew Mecca, Jeewoo Kim, Giusy Caprara, Elizabeth Wagner, Ting-Ting Du, Leonid Petrov, Wenhao Xu, Runjia Cui, Ivan Rebustini, Bechara Kachar, Anthony Peng, and Jung-Bum Shin,
	\emph{Myosin-VIIa is expressed in multiple isoforms and essential for tensioning the hair cell mechanotransduction complex}.
	Nature Communications, 11, Article number: 2066 (2020). \url{https://www.nature.com/articles/s41467-020-15936-z}. 15 pages.
\end{etaremune}



\section*{Students and postdocs}

\begin{enumerate}
    \item \href{https://math.virginia.edu/people/smb2tp/}{Yizhen Li}, UVA Ph.D. student 2024-29
    \item \href{https://mtikhonov.com}{Mikhail Tikhonov}, UVA Ph.D. student 2020-26
    \item \href{https://danielslonim.github.io}{Daniel Slonim}, UVA postdoc 2022-24 (to be Tenure Track at Hillsdale College)
    \item \href{https://math.mit.edu/directory/profile.html?pid=2588}{Svetlana Gavrilova},  HSE bachelor thesis, 2020-21 (now Ph.D. student at MIT)
    \item \href{https://sites.google.com/view/axelsaenz}{Axel Saenz}, UVA postdoc 2016-19 (now Tenure Track at Oregon State)
\end{enumerate}

\section*{Recent organization and service}

\begin{itemize}
	\item [2026:]
	\href{https://www.coinflippers.org/}{Coin Flippers 2026: The ICM Satellite Edition},
	University of Delaware,
	Newark, DE,
	July 21-22, 2026.

	\item [2026:]
	\href{https://jointmathematicsmeetings.org/jmm}{AMS Special Session on Random Tilings, Random Permutations, and Particle Systems},
	Joint Mathematics Meetings,
	Washington, DC,
	January 4-7, 2026.

	\item [2025:]
	Section at the
	\href{https://www.math.miami.edu/mca/}{Mathematical Congress of the Americas 2025},
	University of Miami,
	Miami, FL,
	July 21-25, 2025.

	\item [2025:]
	\href{https://aimath.org/pastworkshops/roadtokpz.html}{AIM workshop
	``All roads to the KPZ universality class''},
	AIM, Caltech, Pasadena, CA, March 17-21, 2025.

	\item [2024:]
	\href{https://math.virginia.edu/2024/09/BlueRidgeProb/}{Blue Ridge Probability Day at University of Virginia},
	October 4, 2024.

	\item [2024:]
	\href{https://lpetrov.cc/vipss2024/}{Virginia Integrable Probability Summer School 2024},
	July 8-19, 2024.

	\item [2024:]
	\href{https://math.virginia.edu/random-lie-2024/}{Workshop ``Randomness and Lie-Theoretic Structures at University of Virginia''},
	March 4-5, 2024.

	\item [2024:]
\href{https://www.jointmathematicsmeetings.org/meetings/national/jmm2024/2300_program_ss43.html}{AMS-AWM Special Session on Solvable Lattice Models and their Applications Associated with the Noether Lecture at the Joint Mathematics Meetings 2024},
San Francisco, CA, January 3-6, 2024.

\item [2023:]
\href{https://sites.google.com/view/sepc2023ii/sepc-2023-ii}{SouthEastern Probability Conference - II at University of Virginia},
August 14-15, Charlottesville, VA, 2023.

\item [2014-current:]
\href{http://math.virginia.edu/seminars/probability/}{University of Virginia Probability Seminar}

% \item [2020, 2022:]
% (\textbf{postponed then canceled due to COVID-19})
% \href{http://www.ams.org/meetings/sectional/2273_program.html}{Special Session on Integrable Probability at the 2020 AMS Spring Southeastern Sectional Meeting at University of Virginia},
% March 13-15, 2020; March 11-13, 2022.

% \item [2021:]
% \href{https://www.matrix-inst.org.au/events/integrability-and-combinatorics-at-finite-temperature/}{Program "Integrability and combinatorics at finite temperature" at MATRIX Institute, Australia (virtual)},
% June 7-18, 2021.

% \item [2020:]
% \href{http://mtikhonov.com/smisp/}{Online conference on Statistical Mechanics, Integrable Systems and Probability},
% April 27 - May 1, 2020.

% \item [2020:]
% (\textbf{postponed due to COVID-19})
% \href{https://www.matrix-inst.org.au/events/integrability-and-combinatorics-at-finite-temperature/}{Program "Integrability and combinatorics at finite temperature" at MATRIX Institute, Australia},
% June 1-19, 2020.

% \item [2019:]
% \href{http://vipss.int-prob.org/}{Virginia Integrable Probability Summer School 2019},
% May 17 - June 8, 2019.

% \item[2018-19:]
% \href{https://lpetrov.cc/reading-2019/}{Reading seminar on Integrable Probability}.

% \item [2018:]
% \href{http://math.virginia.edu/ims/workshop-fall-2018/}{Workshop on Representation Theory, Combinatorics, and Geometry at University of Virginia},
% October 19-21, 2018.

% \item [2018:]
% \href{http://frg.int-prob.org/conference2018/}{Conference "Integrable Probability Boston 2018 (IntProb Boston)" at MIT},
% May 14-18, 2018.

% \item [2017+:]
% \href{http://frg.int-prob.org/}{Developer of the website and forum for the FRG "Integrable Probability"}.

% \item [2017+:]
% \href{http://math.virginia.edu/}{Developer of the University of Virginia Math Department website}.

% \item [2017:]
% \href{http://faculty.virginia.edu/ssp17/}{Conference "Seminar on Stochastic Processes 2017" at University of Virginia},
% March 8-11, 2017.

% \item [2016-17:]
% \href{https://lpetrov.cc/2016/12/reading-seminar/}{Reading seminar on Integrable Probability}.

% \item [2014-17:]
% \href{http://math.virginia.edu/seminars/mathclub/}{Undergraduate Math Club at the University of Virginia}.

\end{itemize}


\section*{Teaching}

\subsection*{University of Virginia (since 2014)}

Complex Variables;
Introduction to Probability;
Introduction to Stochastic Processes;
Calculus III;
Asymptotic representation theory (graduate topics course);
Particle Systems (graduate topics course);
Random matrices (graduate topics course);
Real Analysis and Linear Spaces (graduate).

\subsection*{Northeastern University (2011--2014)}

Calculus II for Sci\&Eng;
Probability and Statistics;
Statistics and Stochastic Processes;
Probability 1 (graduate course);
Topics in Probability (graduate topics course).

\section*{Other service}

\begin{itemize}
	\item
	I am broadening access to AI tools for working mathematicians by \href{https://lpetrov.cc/AI-math/}{sharing} \href{https://storage.lpetrov.cc/research_files/talks/AI_UVA_Oct10.pdf}{best practices} and participating in panel discussions. I also serve as an AI guide for Department of Mathematics, University of Virginia.

	\item
	Member of the editorial boards at ``\href{https://www.springer.com/journal/11040}{\textbf{Mathematical Physics, Analysis and Geometry}}'', ``\href{https://escholarship.org/uc/combinatorial_theory/}{\textbf{Combinatorial Theory}}'', and ``\href{https://imstat.org/journals-and-publications/electronic-journal-of-probability/}{\textbf{Electronic Journal/Communications of Probability}}''.
	\item Program committee member for FPSAC (Formal Power Series and Algebraic Combinatorics),
		2017, 2021, and 2024.
	\item
	I regularly referee scholarly journal papers submitted to numerous journals,
	including
	Ann. Prob., Adv. Math., Adv. Appl. Math., Comm. Math. Phys., Intern. J. Math.,
	Arkiv f\"or Mat., SIGMA, J. Alg. Comb., Comm. Pure Appl. Math., Intern. Math.
	Res. Notices, J. Appl. Probab., J. Comb. Theory A, Symp. Th. Aspects of Comp.
	Sci., J. of Stat. Physics, Electron. Comm. Probab.
	\item
	I am a regular
	reviewer for the Mathematical Reviews database.
	\item
	I reviewed grant proposals and served on panels for several funding agencies.
	\item Broadening access to AI tools for working mathematicians by sharing \href{https://lpetrov.cc/AI-math/}{best practices} and participating in panel discussions (2023).
\end{itemize}


\bigskip

% Footer
\begin{center}
	\begin{footnotesize}
		Last updated: \today \\ \href{\footerlink}{\url{\footerlink}}
	\end{footnotesize}
\end{center}

\end{document}
