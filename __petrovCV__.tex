\documentclass[letterpaper,11pt]{article}
\usepackage{hyperref}
\usepackage{geometry,amsmath,amsfonts}
\usepackage[T1]{fontenc}
\usepackage[sc,osf]{mathpazo}
\usepackage{etaremune,enumerate}
 
\def\name{Leonid Petrov}
\def\footerlink{http://faculty.virginia.edu/petrov/}
\allowdisplaybreaks
\usepackage{longtable}

\hypersetup{
  colorlinks = true,
  urlcolor = black,
  pdfauthor = {\name},
  pdftitle = {\name: Leonid Petrov},
  pdfsubject = {Leonid Petrov},
  pdfpagemode = UseNone
}

\geometry{
  body={6.2in, 8.25in},
  left=1.2in,
  top=1.45in
}

% Customize page headers
\pagestyle{myheadings}
\markright{\name}
\thispagestyle{empty}

% Custom section fonts
\usepackage{sectsty}
\sectionfont{\rmfamily\mdseries\Large}
\subsectionfont{\rmfamily\mdseries\itshape\large}

% Other possible font commands include:
% \ttfamily for teletype,
% \sffamily for sans serif,
% \bfseries for bold,
% \scshape for small caps,
% \normalsize, \large, \Large, \LARGE sizes.

% Don't indent paragraphs.
\setlength\parindent{0em}

% Make lists without bullets
% \renewenvironment{itemize}{
%   \begin{list}{}{
%     \setlength{\leftmargin}{1.5em}
%   }
% }{
%   \end{list}
% }

\begin{document}

{\huge \name}

% Alternatively, print name centered and bold:
%\centerline{\huge \bf \name}

\vspace{0.25in}

\begin{minipage}{0.45\linewidth}
  Department of Mathematics\\
	University of Virginia\\
	141 Cabell Drive, Kerchof Hall\\
	P.O. Box 400137\\
	Charlottesville, VA 22904-4137
\end{minipage}
\begin{minipage}{0.45\linewidth}
  \begin{tabular}{ll}
    Office Phone: & +1-434-924-4167 \\
    Email: & 
    \href{mailto:petrov@virginia.edu}{\tt petrov@virginia.edu}\\&
    \href{mailto:lenia.petrov@gmail.com}{\tt lenia.petrov@gmail.com}
    \\
    Homepage: & \url{http://faculty.virginia.edu/petrov/} \\
  \end{tabular}
\end{minipage}

\section*{Research interests}

Probability, Mathematical Physics, Algebraic Combinatorics, Representation Theory.

\section*{Education}

\begin{itemize}
  \item[2007--2010:]
  Ph.D. studies,\\Institute for Information Transmission Problems (Moscow, Russia).
  \\
  Advisor: \href{http://www.iitp.ru/en/userpages/88/}{Grigori Olshanski}.\\
  Thesis ``Markov Chains on Partitions and Infinite--Dimensional Diffusion Processes''.
  % ,\\
  % defended
  % June 21, 2010.

  \item[2002--2007:]
  Diploma with excellence,\\
  Lomonosov Moscow State University (Russia),\\ Department of Mathematics and Mechanics, Chair of Probability.

  \item 
  [1997--2002:] Moscow High School No. 2.

\end{itemize}

\section*{Employment}

\begin{itemize}
\item[since 2014:]
Assistant Professor\\
at Department of Mathematics, 
University of Virginia, 
Charlottesville, VA, USA.

\item[2011--2014:]
Research Instructor\\ at
Department of Mathematics,
Northeastern University, Boston, MA, USA.

\item[since 2009:]
Research associate\\
at
Dobrushin Mathematics Laboratory,
Institute for Information Transmission Problems, Moscow, Russia (on leave since 2011).
\end{itemize}

\section*{Scholarships/prizes/funding}
\begin{itemize}
  \item[2015:] Prize of the Moscow Mathematical Society.

  \item[2014--2015:] EDF Fellowship of the University of Virginia. 

  \item[2014:] AMS/NSF Travel Grant Award for ICM 2014.

  \item[2011--2013:] RFBR--CNRS grant 11-01-93105 ``Representation theory and noncommutative geometry''.
  
  \item[2010--2012:] RFBR--CNRS grant 10-01-93114 ``New models of Markov processes on point configurations. Applications to stochastic queueing networks''.

  \item[2010:] Dynasty foundation fellowship for young scientists.
    
  \item[2010:] Silver prize of The Fourteenth Moebius Contest.

  \item[2009:] Alexander Kuznetsov/Independent University of Moscow graduate student scholarship.

  \item[2005, 2006:] V. Potatin federal scholarship for academic excellence, leadership and creativity.
\end{itemize}

\section*{Publications}

% \subsection*{Papers in preparation}% (working titles)}

% \begin{enumerate}
%   \item (with Christian Gromoll and Mark Meckes)
%     \emph{Gaussian concentration in last passage percolation with general weight distribution}.
%   \item (with Vadim Gorin) 
%     \emph{Homogenization of nonintersecting lattice random walks and 
%     discrete bulk universality}.
%   \item (with Alexei Borodin)
%     \emph{Inhomogeneous vertex models and symmetric rational functions}.
%   \item (with Alexei Borodin)
%     \emph{TASEP-like particle system in continuous inhomogeneous medium: phase transitions and Tracy-Widom asymptotics}.
% \end{enumerate}

% \subsection*{Publications and preprints}

\begin{etaremune}
    \renewcommand{\labelenumi}{[\theenumi]}
    \item (with Konstantin Matveev)
    \emph{$q$-randomized Robinson--Schensted--Knuth correspondences and random polymers}
    (2015),
    arXiv:1504.00666 [math.PR]. Submitted.%
    \item (with Ivan Corwin)
    \emph{Stochastic higher spin vertex models on the line}
    (2015),
    To appear in 
    Comm. Math. Phys.
    DOI: 10.1007/s00220-015-2479-5,
    arXiv:1502.07374 [math.PR].
    \item (with Alexei Borodin, Ivan Corwin, and Tomohiro Sasamoto)
    \emph{Spectral theory for interacting particle systems solvable by coordinate Bethe ansatz}, 
    Comm. Math. Phys.
    339 (2015), no. 3, 
    1167--1245,
    DOI: 10.1007/s00220-015-2424-7,
    arXiv:1407.8534 [math-ph].
    \item (with Alexey Bufetov)
    \emph{Law of Large Numbers for Infinite Random Matrices over a Finite Field}, 
    Selecta Math. 21 (2015), no. 4, 
    1271--1338,
    arXiv:1402.1772 [math.PR].
    \item (with Alexei Borodin) 
    \emph{Integrable probability: From representation theory to Macdonald processes}, 
    Probability Surveys 11 (2014), 1--58, arXiv:1310.8007 [math.PR]. 
    \item (with Alexei Borodin, Ivan Corwin, and Tomohiro Sasamoto)
    \emph{Spectral theory for the q-Boson particle system},
    Compositio Mathematica 151 (2015), no. 1, 1--67,
    arXiv:1308.3475 [math-ph]. 
    \item (with Ivan Corwin)
    \emph{The q-PushASEP: A New Integrable Model for Traffic in 1+1 Dimension},
    Journal of Statistical Physics,
    160 (2015), no. 4, 1005--1026,
    arXiv:1308.3124 [math.PR].
    \item (with Alexei Borodin) 
    \emph{Nearest neighbor Markov dynamics on Macdonald processes} (2013), 
    arXiv:1305.5501 [math.PR].
    To appear in Advances in Mathematics.%
    \item \emph{The Boundary of the Gelfand-Tsetlin Graph: New Proof of Borodin-Olshanski's Formula, and its q-analogue} (2012), 
    Moscow Mathematical Journal 14 (2014) no. 1, 121--160,
    arXiv:1208.3443 [math.CO].
    \item \emph{Asymptotics of Uniformly Random Lozenge Tilings of Polygons. Gaussian Free Field} (2012), 
    Annals of Probability 43 (2014), no. 1, 1--43,
    arXiv:1206.5123 [math.PR].
    \item \emph{Asymptotics of Random Lozenge Tilings via Gelfand-Tsetlin Schemes} (2012), 
    Probability Theory and Related Fields 160 (2014), no. 3, 429--487,
    arXiv:1202.3901 [math.PR].
    \item \emph{$\mathfrak{sl}(2)$ Operators and Markov Processes on Branching Graphs},
    Journal of Algebraic Combinatorics 38 (2013), no. 3, 663--720,
    arXiv:1111.3399 [math.CO].
    \item \emph{On Measures on Partitions Arising in Harmonic Analysis for Linear and Projective Characters of the Infinite Symmetric Group} (2011), Proceedings of the international conference ``50 years of IITP'', arXiv:1107.0676 [math.CO].
    \item  \emph{Pfaffian Stochastic Dynamics of Strict Partitions},  Electronic Journal of Probability 16 (2011), 2246--2295, arXiv:1011.3329 [math.PR].
    \item \emph{Random Strict Partitions and Determinantal Point Processes}, Electronic Communications in Probability 15 (2010), 162--175, arXiv:1002.2714 [math.PR].
    \item  \emph{Random Walks on Strict Partitions}, Journal of Mathematical Sciences 168 (2010), no. 3, 437--463, arXiv:0904.1823 [math.PR].  
    \item  \emph{Limit Behavior of Certain Random Walks on Strict Partitions}, Russian Mathematical Surveys 64 (2009), no. 6, 1139--1141.
    \item  \emph{A Two-parameter Family of Infinite-dimensional Diffusions in the Kingman Simplex}, Functional Analysis and Its Applications 43 (2009), no. 4, 279--296, arXiv:0708.1930 [math.PR].
     \item 
    \emph{Asymptotic Behavior of a Certain Collection of Particles on a Line Under Synchronization}, Proceedings of the XXVIII Conference of Young Scientists of Department of Mechanics and Mathematics of the Lomonosov Moscow State University (2006), 152--156, in Russian.
\end{etaremune}

\section*{Talks and conferences}

\subsection*{Seminar talks}

\begin{longtable}{llc}
  % 2015, December
  % & University of Wisconsin--Madison
  % \\\\   
  2015, October
  & Penn/Temple Probability Seminar
  \\\\   

  2015, August
  & Institute for Information Transmission Problems
  (Dobrushin Lab Seminar)
  \\\\ 

  2015, May
  & Imperial College London
  \\\\

  2015, March
  & University of Michigan
  \\\\

  2014, November
  & Virginia Tech\\\\

  2014, September
  & Duke University\\\\

  2014, February
  & MIT\\\\

  2014, January 
  & Purdue University &\hspace{110pt}
  \\& University of British Columbia (2 talks)\\\\
  
  2013, December
  & University of Colorado, Boulder
  \\&Penn State University\\&Worcester Polytechnic Institute
  \\&University of Virginia\\&
  Carnegie Mellon University (2 talks)\\\\
  
  2013, November&
  University of Minnesota (2 talks)\\&
  Penn State University (2 talks)\\\\

  2013, October&
  University of Southern California 
  \\&
  University of California, Los Angeles
  \\&Rutgers University\\\\
  
  2013, June& Institute for Information Transmission Problems
  (Dobrushin Lab Seminar)\\\\

  2013, March& Independent University of Moscow (General Seminar ``Globus'')\\\\

  2013, February & Arizona State University\\\\

  2012, October & Princeton University \\
  & Columbia University\\\\

  2012, September & MIT \\\\ 

  2012, August
  & Institute for Information Transmission Problems
  (Yakov Sinai's Seminar)
  \\\\ 

  2012, March & New York University\\\\

  2011, October & Brown University \\
  &University of Michigan
  \\
  \\
  2011, May& Alfr\'ed R\'enyi Institute of Mathematics
  \\
  & Budapest University of Technology and Economics
  \\\\

  2008, October & 
  St. Petersburg Department of Steklov Mathematical Institute% (Representation Theory and Dynamical Systems) 
  \\
\end{longtable}
\bigskip

In addition to that, I have presented results of my research in a number of talks given at various research seminars in Moscow, Russia (at Moscow State University, Institute for Information Transmission Problems, Independent University of Moscow, etc.) in 2005--2011, at Northeastern University, Boston, MA, USA in 2011--2014, and
at University of Virginia, Charlottesville, VA, USA since 2014.


\subsection*{Conference talks}

\begin{etaremune}

\item Workshop on Classical and Quantum Integrable Systems, 
Institute for High Energy Physics, July 2015, Protvino, Russia.
Tutorial talk \emph{``Integrable probability and Bethe ansatz''}.

\item 
``Random Interfaces and Integrable Probability'' workshop (part of 
``Statistical Mechanics, Integrability and Combinatorics'' program), 
Galileo Galilei Institute (GGI), June 2015, Florence, Italy.

\item ``Random Polymers and Algebraic Combinatorics'' workshop, 
Mathematical Institute of the University of Oxford, May 2015, Oxford, UK.

\item ``Limit shapes'' workshop, ICERM, April 2015, Providence, RI, USA. Video of the talk available 
online at 
\url{http://icerm.brown.edu/video_archive/videos/sp_s15_w3/Solving_interacting_particle_systems_by_Fourier-like_transforms_]_Leonid_Petrov,_University_of_Virginia.php}

\item Central Spring AMS Sectional Meeting at Michigan State University, March 2015, East Lansing, MI, USA. \emph{Invited special session talk}.

\item Columbia--Princeton Probability Day, March 2015, Princeton, NJ, USA.
I was one of the two junior speakers at a one-day seminar series,
with a talk \emph{``Eigenfunctions of stochastic integrable particle systems''}.

\item Inhomogeneous Random Systems conference, Henri Poincar\'e Institute, January 2015, Paris, France.

\item International Congress of Mathematicians, August 2014, 
Seoul, South Korea. \emph{Contributed talk}.

\item Workshop 
``From Macdonald Processes to Hecke Algebras and Quantum Integrable Systems''
at Henri Poincar\'e Institute,
May 2014, Paris, France.

\item 
Workshop
``Random Matrices and Jacobi Operators''
at 
Mittag-Leffler Institute,
May 2014, Stockholm, Sweden.

\item
Columbia / Courant Joint Probability Seminar Series on Kardar-Parisi-Zhang Universality. 
I was one of the three speakers of the seminar series,
with a talk
\emph{``Markov Dynamics on Macdonald Processes''}.
October 2013, New York, NY, USA.

\item
Cornell Probability Summer School, July 2013, Ithaca, NY, USA.
\emph{Tutorial sessions for the lecture course of 
A.~Borodin, and a short talk}.

\item 
Random Tilings Workshop at the 
Simons Center for Geometry and Physics,
February 2013, Stony Brook, NY, USA.

\item  
MSRI ``Random Spatial Processes'' program,
April 2012, Berkeley, CA, USA.


\item 
Interacting Particle Systems, Growth Models, and Random Matrices Workshop at the University of Warwick, March 2012, Warwick, UK.


\item International conference ``50 years of IITP'', July 2011, Moscow, Russia.
  
\item EURANDOM Workshop YEP VIII 2011 ``Stochastic Models for Population Dynamics'', March 2011, Eindhoven, Netherlands.

\item Mathematics -- XXI century. PDMI 70th anniversary, September 2010, St.~Petersburg, Russia.

\item PIMS/UBC School in Probability, June 2009, UBC, Vancouver, Canada. 

\item 
Summer School ``Large N Limits'', August 2008, Bitche, France. 
\end{etaremune}

\subsection*{Other events participated}

\begin{etaremune}
\item 37th Midwest Probability Colloquium, Northwestern University, 
October 2015,
Evanston, IL, USA.

\item Workshop ``Analytic Tools in Probability and Applications'', IMA, April 2015, Minneapolis, MN, USA.

\item Seminar on Stochastic Processes 2015 (\emph{with participation in a panel discussion}), 
April 2015, University of Delaware, 
Newark, DE, USA.

\item Workshop on Moduli Spaces, Derived Geometry, and Geometric Representation Theory, November 2014, University of North Carolina at Chapel Hill, NC, USA.

\item 2014 Charles River Lectures on Probability Theory and Related Topics, 
October 2014, Harvard University, Boston, MA, USA.

\item Workshop ``Stochastic Analysis: Around the KPZ Universality Class'' at Mathematisches Forschungsinstitut Oberwolfach,
June 2014, Oberwolfach, Germany.

\item 2013 Charles River Lectures on Probability Theory and Related Topics, 
October 2013, MIT, Boston, MA, USA.

\item Summer School ``KPZ equation and rough paths'', 
June 2013, 
Lebesgue Center of Mathematics, Rennes, France.

\item 2012 Charles River Lectures on Probability Theory and Related Topics, 
October 2012, Microsoft Research, Boston, MA, USA.

\item St. Petersburg School in Probability and Statistical Physics, June 2012, St. Petersburg, Russia.

\item Algebraic Geometry Northeastern Series (AGNES) Workshop, 
October 2011,
Stony Brook, NY, USA.

\item Clay Institute Summer School ``Probability and Statistical Physics in Two and more Dimensions'', July 2010, Buzios, Brazil. 

\item Midrasha Mathematicae: The Mathematics of Oded Schramm, December 2010, Hebrew University, Jerusalem, Israel. 

\item Summer School ``Structures in Lie Representation Theory'', August 2009, Jacobs University, Bremen, Germany. 

\item PROMYS summer program at Boston University for high school students, Boston, MA, USA, 2001.
\end{etaremune}

\section*{Teaching}

\begin{itemize}
  \item[Fall 2015:]   
  University of Virginia. \\
  MATH 3100 --- Introduction to Probability.

	\item[Spring 2015:]   
	University of Virginia. \\
	MATH 5110 --- Introduction to Stochastic Processes.

	\item[Fall 2014:]   
	University of Virginia. \\
	MATH 3100 --- Introduction to Probability.

  \item[Spring 2014:]   
  Northeastern University. \\
  MATH 7241 --- 
  Probability 1 (graduate).

  \item[Fall 2013:]   
  Northeastern University. \\
  MATH 3081 --- 
  Probability and Statitics, 2 sections.
  \item[Spring 2013:] 
  Northeastern University. 
  \\
  MATH 4581 --- Statistics and Stochastic Processes.
  \item[Fall 2012:]
  Northeastern University. 
  \\
  MATH 1342 --- 
  Calculus II for Sci\&Eng (honors section). Part of a novel Department of Mathematics' teaching experiment with ``inverted'' sections: the instructor assigns watching video-lectures accompanying the textbook for homework, and spends class-time going over problems and clarifying the material.

  MATH 7382 --- Topics in Probability (graduate): an expository introductory-level graduate course on solvable probabilistic models, including tools of algebraic combinatorics and representation theory.

  \item[Spring 2012:] 
  Northeastern University. \\
  MATH 3081 --- 
  Probability and Statitics, 2 sections.
  \item[Fall 2011:]
  Northeastern University.\\
  MATH 1342 --- 
  Calculus II for Sci\&Eng.
  \item[Spring 2011:]
  ``Math in Moscow'' programme in English for international students at the Independent University of Moscow. A course in Combinatorics.

  \item[2007---2008:]
  Moscow High School No. 17.
  Teacher of mathematics.
\end{itemize}

\section*{Other}

I co-organize the
Probability Seminar and the Undergraduate Math Club
at the University of Virginia.

% Prepared lecture notes for the courses:
% \begin{enumerate}
%   \item A number of 
%   undergraduate mathematical courses 
%   at the Moscow State University, 
%   including advanced probability courses
%   by A. Bulinski (2003--2006).

%   \item ``Statistical Mechanics and the Renormalisation Group''taught by D. Brydges
%   at the PIMS-UBC Summer School 2009.
%   Available at \\
%   \url{http://www.math.ubc.ca/~db5d/SummerSchool09/lectures-db/smrg.pdf}.

%   \item ``Asymptotic representation theory'',
%   taught by G. Olshanski 
%   at the Independent University of Moscow,
%   Fall 2009 and Spring 2010 (in Russian). 
%   Available at 
%   \url{http://ium.mccme.ru/postscript/f09/representation.zip},
%   \url{http://ium.mccme.ru/postscript/s10/representation2.pdf}.

%   \item ``Topics in probability''
%   course I taught
%   at the Northeastern University, 
%   Fall 2012. Available at
%   my homepage:
%   \url{http://faculty.virginia.edu/petrov/7382F12/LectureNotes.pdf}.

%   \item ``Integrable probability: From representation theory to Macdonald processes''
%   taught by A. Borodin (with tutorial sessions 
%   given by myself)
%   at the Cornell Probability Summer School 2013.
%   See publication [15].
% \end{enumerate}

\smallskip


I regularly referee scholarly journal papers
	submitted to journals such as Comm. Math. Phys., 
	Ann. Prob., Jour. Alg. Combinatorics,
	Comm. Pure Appl. Math.,
	Intern. Math. Res. Notices,
	Jour. Appl. Probab.,
	Jour. Comb. Theory A,
	Jour. of Stat. Physics,
	Electron. Comm. Probab.,
	and 
	Symposium on Theoretical Aspects of Computer Science.
	Also I am a regular reviewer for the
	Mathematical Reviews database.

% Referee works submitted to various contests of the Independent University of Moscow (in particular, Moebius Contest, \url{http://www.moebiuscontest.ru/index-en.html}). Act as opponent on diploma thesis defences at the IUM.

% \smallskip

% For several years maintained web databases of mathematical seminar announcements at the Institute for Information Transmission Problems (Moscow, Russia).

% \smallskip

% My profile on MathOverflow: \url{http://www.mathoverflow.net/users/979/leonid-petrov}

\smallskip

Informatic skills : 
\LaTeX, 
\texttt{Mathematica}, 
Computer experimentation,
Probabilistic visualization,
\texttt{git}, \texttt{SVN},
\texttt{Python}, \texttt{C/C++}, \texttt{Qt},
\texttt{html}, \texttt{UNIX}.

\section*{Personal}

\textbf{Date of birth:} November 5, 1985

\textbf{Languages:} English (fluent), Russian (native)

\textbf{Citizenship:} Russia

\textbf{Phone:} +1-857-210-7948

\textbf{Skype:} \texttt{lenis2000}


\bigskip

% Footer
\begin{center}
  \begin{footnotesize}
    Last updated: \today \\
    \href{\footerlink}{\url{\footerlink}}
  \end{footnotesize}
\end{center}
 
\end{document}